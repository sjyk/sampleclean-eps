\section{Background}

\subsection{Entity Resolution}
The entity resolution (ER) problem [cite] is the study of linking database records to real-world entities.
While traditionally, the operation only discusses the final step of duplicate record linkage [cite], there are often
many pre-processing steps such as string processing and imputation that are often run before record linkage.

We formalize this problem as a composition of two functional operations: deduplication and correction.
\begin{definition} Entity Resolution. 
An Entity Resolution task (ER) for a relation $R$ is a composition of the 
following deterministic operations:
\begin{itemize}
\item Correct $\kappa(R_i,a_{1})$ which for each record in $R_i$ returns a new record with attribute $a_{1}$ modified.
\item Deduplication $\delta(R_i)$ for each record in $R_i$ returns a new record such that the new record is contained in $R_i$.
\end{itemize}
%We constrain that $\kappa$ and $\delta$ are consistent; that is given two records that are identical on the projection their operation is identical.
\end{definition}

To make this more concrete, consider the following scenario. 
Suppose we want to count the number of Mediterranean restaurants in the city of Berkeley, and 
our dataset is of the following schema: \textbf{Restaurant(Name, Type, City)}.
This dataset contains duplicates (two records that point to the same real world entity) and missing type labels.
To fill in the missing type labels we can train a classifier on those labels that do exist and predict a label from the features of the restaurant name $\kappa$.
The operation that merges two records that refer to the same restaurant is a $\delta$ operation.

\subsection{Transitivity Errors Are Common When Using Heuristics}
Let us explore one type of error that can happen when $\delta$ and $\kappa$ are heuristics.
We can present a simplified problem to better understand the challenges.
A special case of the general ER problem posed above is ER on a string-valued single attribute.

The first problem is a false-negative problem.
We may find that string $a$ merges to string $b$, and $b$ merges to $c$. 
This would imply a linkage between $a$ and $c$; which is missing.
Our algorithmic challenge then is to find as many of these missing edges.
We argue that some of these edges can be discovered by making the graph more dense 
by also looking at the prior history of transformations that happened to the string (which we will discuss in the next section).

Conversely, another problem that can happen is spurious edges.
$a$ and $b$ are linked when in reality they are two separate entities.
In this case, the random permutations find and more highly weight those edges that are invariant to the permutation of operations.

\subsection{SampleCleanPipeline Basic API}
In SampleClean [cite], we proposed using sampling to scale data cleaning to large datasets.
Instead of cleaning an entire dataset, we applied an expensive data cleaning algorithm to a sample, then 
answer queries using the clean sample.
This allowed for previously prohibitively expensive techniques (eg. crowdsourcing) to still estimate results
on large datasets.

We implemented SampleClean as an application on the BDAS stack.
This application contains a library of data cleaning techniques, and an API to specify new data cleaning algorithms.
The main component of this library is the \scp class.
This class abstracts the management of execution of data cleaning operations away from the user.

Users define an ordered collection of data cleaning algorithms e.g. time format cleaning, nearest neighbor imputation, and Jaccard similarity deduplication.
Then, they define a series of hints of which constrain possible orders
of the data cleaning algorithms, e.g. run Algo1 before Algo2, run Algo2 last.
As in the examples, hints can be of the form of partial orders and absolute indices.

Once the pipeline is instantiated, there is an \textbf{optimize()} method which
translates the unordered collection and hints into an execution plan.
Then the \textbf{exec(sampleName)} method applies the pipeline to a sample.

\begin{lstlisting}
SampleCleanPipeline( 
  ops: Collection[SampleCleanAlgorithm],
  hints: List[Hints] 
) 


sample_clean_pipeline.optimize()

sample_clean_pipeline.exec("my_sample")
\end{lstlisting}

In this paper, we explore the final two steps, \textbf{optimize()} and \textbf{exec(sampleName)}, 
for a limited set of operators (ER operators).

The goal of SampleClean was to make data cleaning happen at interactive latencies. 
If EPS takes a long time to run this defeats the purpose of the system.
We also extend the API to the streaming case where partial results can stream back to the user.

\begin{lstlisting}
sample_clean_pipeline.stream_exec(
  "my_sample", 
  optimizer,
  onChangeCallback)
\end{lstlisting}

\subsubsection{Problem Statement}
We are given a collection of ER operators and hints that define partial orders and absolute indices.
The EPS algorithm has to output two things: (1) A data structure $T$ which represents an optimized plan (which call a \emph{proposal}), and
(2) translate $T$ into actual changes to the tuples in the sample.
For the streaming case, we define an incomplete optimization structure $T_k$ and an execution engine which updates its executions when the data
structure is updated $T_{k+1}\leftarrow T_k$. 











